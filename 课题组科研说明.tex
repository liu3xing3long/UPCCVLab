\documentclass[a4paper,12pt]{article}
\usepackage{styles/iplouccfg}
\usepackage{styles/zhfontcfg}
\hypersetup{%pdfpagemode=FullScreen,%
            pdfauthor={Haiyong Zheng},%
            pdftitle={Title},%
            CJKbookmarks=true,%
            bookmarksnumbered=true,%
            bookmarksopen=false,%
            plainpages=false,%
            colorlinks=true,%
            citecolor=green,%
            filecolor=magenta,%
            linkcolor=cyan,%red(default)
            urlcolor=cyan}


\title{课题组科研说明 V1.0} %标题
\author{宫延河}
\date{2015年05月17日} %日期,不加默认为当前日期

\begin{document}

\maketitle

%\section{安全与健康}
%
%\begin{enumerate}
%\item 无论什么时候什么情况下,安全与健康永远都是第一位的!
%\item 我们崇尚快乐科研,高效工作!
%\item 优秀离不开坚持不懈专注用心的努力,但安全和健康要比任何事都重要,以下要求请务必在保证安全和身体健康的前提下进行!
%\end{enumerate}

\section{目标}

\begin{enumerate}

\item 打造顶尖的研究团队

\item 从事顶尖的研究工作

\item 建立顶尖的人脉网络

\end{enumerate}

\section{科研时间}

\begin{description}

\item[上午] 8:30--11:50之间满3小时

\item[下午] 2:30-5:50之间满3小时

\item[晚上] 6:30--10:30之间满3小时

\end{description}

\begin{itemize}

\item 请确保每天工作9个小时、每周至少工作6天,以保证每周工作时间能在{\color{red}{50小时}}以上。

\item 实验室暑假可以自由调休2周,寒假在一个月以内。

\item 如果超过半天(包括半天)不能到实验室工作或每周工作时间达不到50小时(有课时,上课时间按工作时间计算),须与导师进行沟通说明。

\item 但无论怎样,都不能拿出休息的时间来工作;当你觉得你不得不这样做的时候,你应该首先考虑自己是否做到了合理高效的利用时间。

\end{itemize}

\section{工作要求}

\begin{itemize}

\item 实验室不允许做与学习、科研等无关的事情,尤其浏览无关的网站和视频。

\item 以坚持、专注、用心、积极、进取的态度全身心的投入到科研工作中。

\end{itemize}

\section{研究方向}

\begin{enumerate}

\item 三维模型检索

\item 服装商品图像检索

\item 智能视频分析

\item 机器学习

\item 模式识别

\end{enumerate}

\section{科研要求与国内外学术交流}

\begin{itemize}

\item 对团队有高度的责任心,以开放、诚恳、学习的心态与团队成员交流、学习。

\item 有任何想法或问题都要及时跟老师和团队进行沟通。

\item 目前实验室在未来 3 年有比较充足的资金支持大家从事科研、交流(国内和国际研讨与会议)、发表等工作。

\item 鼓励向计算机视觉相关国际顶级会议投稿并到国外参加会议!

\item 我们不为发论文而发论文,追求卓越的论文质量而不是粗制滥造的数量

\end{itemize}

目前我们主要关注的计算机视觉、机器学习和模式识别国际会议:

\begin{itemize}

\item 一级会议:ICCV、CVPR、ECCV、ICML

\item 二级会议:BMVC、ICIP、ICPR、ACCV

\end{itemize}

期刊:

\begin{itemize}

\item 国际期刊:PAMI、IJCV、PR、TIP

\item 国内期刊:计算机学报、自动化学报、电子学报、中国科学F辑、中国图象图形学报

\end{itemize}

\section{科研安排}

以下工作须在与导师不断的沟通下完成:

\begin{enumerate}

\item 研究生一年级第一学期的前两个月:熟悉必要的科研工具,学习必要的基础知识(图像处理相关),进行必要的科研训练。

\item 研究生一年级第一学期的后两个月;查阅方向相关的科研书籍与学术论文,明确各自的科研方向(大)。

\item 研究生一年级第二学期:根据各自的科研方向,结合实验室的科研项目,寻找具体可行的科研方向(小且细)
。

\item 研究生二年级第一学期(包括研一第二学期末暑假):根据所明确的小且细的科研方向,不断实验(实践)、反复思考、频繁讨论,进行深入的科研工作,撰写相关学术论文并投稿(国际会议)。

\item 研究生二年级第二学期(包括研二第二学期末暑假):根据前期工作进一步完善,进一步撰写相关学术论文并投稿(国际期刊)。

\item 研究生三年级第一学期:根据前面两年的工作积累,构思硕士毕业论文,并进行必要的工作补充。

\item 研究生三年级第二学期:硕士毕业论文的撰写、答辩等工作。

\end{enumerate}

\section{科研工具}

\begin{description}

\item[操作系统] UNIX/Linux、Windows 7/8 64bit

\item[编译工具] CMake 3.2

\item[编程语言] Matlab 2013a及以上与C/C++(VC6.0 OpenCV1.0 \& 2.4.*) VS2010/VS2012(for DeepLearning)

\item[并行加速工具] CUDA6.5

\item[脚本工具] python(x,y) 2.7.9

\item[文档排版工具] \LaTeX \& WinEdt8

\item[版本控制系统] Git/GitHub

\item[文献整理工具] EndNote X7

\end{description}

\section{基础学习}

\begin{description}

\item[入门]中科院自动化所献给CV和CG入门者之科研经验浅问细答兼与大家探讨 祥见群共享

\item[入门]《学术知识普及》祥见群共享

\item[入门] 研究所新生完全求生手册 祥见群共享

\item[基础]《CV之路》祥见群共享

\item[基础]《数学之美》、浪潮之巅 祥见群共享



\item[资料备份] \href{https://github.com/imistyrain/UPCRules}{石大可视化课题组程序归档规范}

\end{description}

%\section{MOOC学习}
%
%\begin{itemize}
%
%\item 从目前的一年级(包括一年级)开始,以后所有进入实验室的同学除了完成学校要求的研究生培养计划外,还必须完成实验室指定课程(MOOC课程或“慕课”)的学习并务必在一年级结束前获得学分证书后方可最后毕业。
%
%\item 由于网上MOOC课程可能会有所更新和变更等,所以每学年指定课程可能不尽相同,但总体来说会包括科研、程序、数学和专业四部分,其中数学和专业对于不同研究方向的同学会略有所不同【目前我感兴趣的研究方向包括计算机视觉和生物信息学,这两个方向在数学和专业上的要求也略有不同】。
%
%\item 2013级和2014级(保研)必修课程包括:
%
%\begin{enumerate}
%
%\item 中国科学技术大学《文献管理与信息分析》:\url{http://mooc.ustc.edu.cn}
%\item 北京大学《数据结构与算法》:\url{https://www.coursera.org/course/dsalgo} (该课程需要先修知识《计算概论》:\url{https://www.coursera.org/course/pkuic},可选修但不要求必须获得学分证书。)
%\item 计算机视觉方向加必修《Numerical Analysis》【自学+每周汇报与讨论的方式】
%\item 生物信息学方向加必修《鸟哥的Linux私房菜:基础学习篇和服务器架设篇》(第三版)【自学+每周汇报与讨论的方式】
%\item 生物信息学方向加必修北京大学《生物信息学:导论与方法》:\url{https://www.coursera.org/course/pkubioinfo}
%
%\end{enumerate}
%
%\item 要求:
%
%\end{itemize}
%
%\begin{enumerate}
%
%\item 每周公开课学习后请务必使用英文向全组【邮件列表组地址:\url{zhenglabouc@googlegroups.com}】通过邮件发表个人总结、观点、收获、感想等。
%\item 按照每门课程的要求完成课程学习并获得相应的学分证书,如果第一次未通过或错过选课时间等,请下学期继续注册学习直至获得学分证书,并务必在研究生一年级结束前修得学分获得证书。
%
%\end{enumerate}

\section{英语学习}

\begin{description}

\item[读] 除了必要的英文论文阅读外,请每天拿出一定时间(半个小时到一个小时)来阅读感兴趣的英语杂志等。

\item[写] 每周进行周小结,总结一周内的进展;必须通过邮件、文档等尽力进行英文写作训练,只读不写无法有真正的提高。

\item[听] 在非工作时间可以讨论观看励志、科技等方面的英文视频,工作时间可以观看\href{http://vision.ouc.edu.cn/valse/}{VALSE}和\href{http://techtalks.tv/}{techtalk}上的技术视频

\item[说] 每学期务必至少有一次口头报告,讲述自己在某个领域阅读论文的进展;每位研究生在就读期间争取能够到国外参加一次国际会议,以锻炼与人英语沟通的能力。

\end{description}

\section{公共盘}

\begin{itemize}

\item 公共盘为实验室历年积累的资料,主要包括实验室数据、毕业研究生
的论文和程序等,还有一些经典的视频资料,供大家参考。

\item 新进学生需提供用户名和密码。

\end{itemize}

\section{工作环境}

\begin{itemize}

\item 在经费允许的情况下,实验室会为每位研究生配备科研相关的环境,如计算机、办公用品、打印复印等。

\item 在经费允许的情况下,根据工作情况发放劳务费。

\item 对于研究生期间的优秀成果(高质量会议和期刊论文),实验室会进行不同程度的奖励和鼓励。

\item 其他必要的要求和建议请及时沟通,在必要的情况下实验室会尽力解决。

\end{itemize}

\section{参考资料}

\begin{itemize}
\item[郑海永]\href{http://vision.ouc.edu.cn/~zhenghaiyong/}{中国海洋大学} 特别鸣谢
\end{itemize}
\end{document}
